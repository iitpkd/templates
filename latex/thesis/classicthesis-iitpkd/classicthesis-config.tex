% ****************************************************************************************************
% classicthesis-config.tex
% formerly known as loadpackages.sty, classicthesis-ldpkg.sty, and classicthesis-preamble.sty
% Use it at the beginning of your ClassicThesis.tex, or as a LaTeX Preamble
% in your ClassicThesis.{tex,lyx} with % ****************************************************************************************************
% classicthesis-config.tex
% formerly known as loadpackages.sty, classicthesis-ldpkg.sty, and classicthesis-preamble.sty
% Use it at the beginning of your ClassicThesis.tex, or as a LaTeX Preamble
% in your ClassicThesis.{tex,lyx} with % ****************************************************************************************************
% classicthesis-config.tex
% formerly known as loadpackages.sty, classicthesis-ldpkg.sty, and classicthesis-preamble.sty
% Use it at the beginning of your ClassicThesis.tex, or as a LaTeX Preamble
% in your ClassicThesis.{tex,lyx} with % ****************************************************************************************************
% classicthesis-config.tex
% formerly known as loadpackages.sty, classicthesis-ldpkg.sty, and classicthesis-preamble.sty
% Use it at the beginning of your ClassicThesis.tex, or as a LaTeX Preamble
% in your ClassicThesis.{tex,lyx} with \input{classicthesis-config}
% ****************************************************************************************************
% If you like the classicthesis, then I would appreciate a postcard.
% My address can be found in the file ClassicThesis.pdf. A collection
% of the postcards I received so far is available online at
% http://postcards.miede.de
% ****************************************************************************************************

\RequirePackage{silence} % :-\suppress an unnecessary warning in compilation
    \WarningFilter{scrreprt}{Usage of package `titlesec'}
    \WarningFilter{titlesec}{Non standard sectioning command detected}
    \WarningFilter{hyperref}{Token not allowed in a PDF string (PDFDocEncoding)}
   
% ****************************************************************************************************
% 0. Set the encoding of your files. UTF-8 is the only sensible encoding nowadays. If you can't read
% äöüßáéçèê∂åëæƒÏ€ then change the encoding setting in your editor, not the line below. If your editor
% does not support utf8 use another editor!
% ****************************************************************************************************
\PassOptionsToPackage{utf8}{inputenc}
  \usepackage{inputenc}

\PassOptionsToPackage{T1}{fontenc} % T2A for cyrillics
  \usepackage{fontenc}

\PassOptionsToPackage{final}{microtype}  %jasine
\usepackage[final]{microtype} %jasine

\emergencystretch=1em

% *******************************************************************************************************************
% Note : Do not edit above this line
% Note : Do not edit options to classicthesis given below except the options drafting, printready and numsupervisors
% and eulermath
% 
% *******************************************************************************************************************
\PassOptionsToPackage{
   tocaligned=false, 
  dottedtoc=true,
  eulerchapternumbers=true, 
  linedheaders=false,      
  eulermath=true, %true, %false %recommended value is true. If false is used, cmmodern will be used.
  beramono=true,   
  palatino=true,    
  style=classicthesis, 
  floatperchapter=true,     % numbering per chapter for all floats (i.e., Figure 1.1)
  numsupervisors=one, %one %two %three %set according to your number of supervisors
  %drafting=true,    % print version information on the bottom of the pages only for draft
  drafting=false, % for final version this line is to be used
  printready=false %Enable this for making title page and hyperreflinks in colour for screen reading
  %printready=true  %Enable this for making title page and hyperreflinks in black for printing
 }{classicthesis}


% ****************************************************************************************************
% 2. Personal data and user ad-hoc commands (insert your own data here)
% ****************************************************************************************************

\newcommand{\myTitle}{Main Title of Your Thesis\xspace}
\newcommand{\mySubtitle}{Subtitle of Your Thesis\xspace}
\newcommand{\myDegree}{Doctor of Philosophy\xspace}
%\newcommand{\myDegree}{Master of Science \xspace}
%\newcommand{\myDegree}{Master of Science in Engineering \xspace}
\newcommand{\myName}{Your Full Name\xspace}
\newcommand{\myGender}{her\xspace}
%\newcommand{\myGender}{him\xspace}
\newcommand{\mySupervisorOne}{First Supervisor Name}
\newcommand{\mySupervisorTwo}{Second Supervisor Name}%used only if necessary
\newcommand{\mySupervisorThree}{Third Supervisor Name}%used only if necessary
\newcommand{\myDepartment}{Department of Computer Science and Engineering\xspace}
\newcommand{\myUni}{Indian Institute of Technology Palakkad\xspace}
\newcommand{\myLocation}{Palakkad\xspace}
\newcommand{\myTime}{Month YYYY\xspace}
\newcommand{\myVersion}{\classicthesis}
% ********************************************************************
% Setup, finetuning, and useful commands
% ********************************************************************
\providecommand{\mLyX}{L\kern-.1667em\lower.25em\hbox{Y}\kern-.125emX\@}
\newcommand{\ie}{i.\,e.}
\newcommand{\Ie}{I.\,e.}
\newcommand{\eg}{e.\,g.}
\newcommand{\Eg}{E.\,g.}
% ****************************************************************************************************


% ****************************************************************************************************
% 3. Loading some handy packages
% ****************************************************************************************************
% ********************************************************************
% Packages with options that might require adjustments
% ********************************************************************
\PassOptionsToPackage{ngerman,american}{babel} % change this to your language(s), main language last
\usepackage{babel}

\usepackage{csquotes}
%%%%%You may change the reference format to any of the options given agaist the parameters. 
%%%%%It is suggested that you maintain compatibility with natbib.
\PassOptionsToPackage{%
  %backend=biber,bibencoding=utf8, %instead of bibtex
  backend=bibtex8,bibencoding=ascii,%
  language=auto,%
  style=numeric-comp,%
  %style=authoryear-comp, % Author 1999, 2010
  %bibstyle=authoryear,dashed=false, % dashed: substitute rep. author with ---
  sorting=nyt, % name, year, title
  maxbibnames=10, % default: 3, et al.
  %backref=true,%
  natbib=true % natbib compatibility mode (\citep and \citet still work)
}{biblatex}
\usepackage{biblatex}

\PassOptionsToPackage{fleqn}{amsmath}       % math environments and more by the AMS
   \usepackage{amsmath}

\usepackage{amssymb}
\usepackage{wasysym}
\usepackage{amsthm} 
\usepackage{mathrsfs}
\usepackage{mathtools}
\usepackage{epsfig}
%%%%Theorem like environments follow chapterwise numbering, 
%%%%only exceptions being 'Remark' and 'Note'.
%%%%Do not edit this basic setting, if you are adding more such environments.
\theoremstyle{plain}% default
\newtheorem{theorem}{Theorem}[section]
\newtheorem{lemma}[theorem]{Lemma}
\newtheorem{proposition}[theorem]{Proposition}
\newtheorem{corollary}[theorem]{Corollary}
\newtheorem{property}[theorem]{Property}

\theoremstyle{definition}
\newtheorem{definition}[theorem]{Definition}
\newtheorem{conjecture}[theorem]{Conjecture}
\newtheorem{example}[theorem]{Example}
\newtheorem{observation}[theorem]{Observation}

\theoremstyle{remark}
\newtheorem*{remark}{Remark}
\newtheorem*{note}{Note}
\newtheorem{claim}{Claim}[chapter]

\usepackage[ruled,vlined,algochapter]{algorithm2e}

% ********************************************************************
% General useful packages
% ********************************************************************
\usepackage{graphicx} %
\usepackage{scrhack} % fix warnings when using KOMA with listings package
\usepackage{xspace} % to get the spacing after macros right
\PassOptionsToPackage{printonlyused,smaller}{acronym}
  \usepackage{acronym} % nice macros for handling all acronyms in the thesis
  \def\bflabel#1{{\acsfont{#1}\hfill}}
  \def\aclabelfont#1{\acsfont{#1}}
% ****************************************************************************************************
%\usepackage{pgfplots} % External TikZ/PGF support (thanks to Andreas Nautsch)
%\usetikzlibrary{external}
%\tikzexternalize[mode=list and make, prefix=ext-tikz/]
% ****************************************************************************************************




% ****************************************************************************************************
% 4. Setup floats: tables, (sub)figures, and captions
%%%Do not edit this setting
% ****************************************************************************************************
\usepackage{tabularx} % better tables
  \setlength{\extrarowheight}{3pt} % increase table row height
\newcommand{\tableheadline}[1]{\multicolumn{1}{l}{\spacedlowsmallcaps{#1}}}
\newcommand{\myfloatalign}{\centering} % to be used with each float for alignment
\usepackage{subfig}
% ****************************************************************************************************


% ****************************************************************************************************
% 5. Setup code listings
%%Do not edit this setting
% ****************************************************************************************************
\usepackage{listings}
%\lstset{emph={trueIndex,root},emphstyle=\color{BlueViolet}}%\underbar} % for special keywords
\lstset{language=[LaTeX]Tex,%C++,
  morekeywords={PassOptionsToPackage,selectlanguage},
  keywordstyle=\color{CTkeyword},%\bfseries,
  basicstyle=\small\ttfamily,
  %identifierstyle=\color{NavyBlue},
  commentstyle=\color{CTcomment}\ttfamily,
  stringstyle=\rmfamily,
  numbers=none,%left,%
  numberstyle=\scriptsize,%\tiny
  stepnumber=5,
  numbersep=8pt,
  showstringspaces=false,
  breaklines=true,
  %frameround=ftff,
  %frame=single,
  belowcaptionskip=.75\baselineskip
  %frame=L
}
% ****************************************************************************************************

%%%%Note that for list of symbols to be inluded the following commands needs to be executed
%%%%pdflatex ClassicThesis.tex
%%%%makeindex ClassicThesis.nlo -s nomencl.ist -o ClassicThesis.nls
%%%%pdflatex ClassicThesis.tex
\usepackage{nomencl}
\makenomenclature
\renewcommand{\nomname}{List of Symbols}
% ****************************************************************************************************
% 6. Last calls before the bar closes
% ****************************************************************************************************
% ********************************************************************
% Her Majesty herself
% ********************************************************************
\usepackage{classicthesis}


% ********************************************************************
% Fine-tune hyperreferences (hyperref should be called last)
%%%Do not edit this setting
% ********************************************************************
\hypersetup{%
  colorlinks=true, linktocpage=true, pdfstartpage=3, pdfstartview=FitV,%
  breaklinks=true, pageanchor=true,%
  pdfpagemode=UseNone, %
  plainpages=false, bookmarksnumbered, bookmarksopen=true, bookmarksopenlevel=1,%
  hypertexnames=true, pdfhighlight=/O,%
  urlcolor=CTurl, linkcolor=CTlink, citecolor=CTcitation, %
  pdftitle={\myTitle},%
  pdfauthor={\textcopyright\ \myName, \myUni},%
  pdfsubject={},%
  pdfkeywords={},%
  pdfcreator={pdfLaTeX},%
  pdfproducer={LaTeX with hyperref and classicthesis}%
}

%%%%%%%%%Do not edit below this line %%%%%%%%%%%%%%%%%%%%%%%%%%%%%%%%%

% ********************************************************************
% Setup autoreferences (hyperref and babel)
% ********************************************************************
% There are some issues regarding autorefnames
% http://www.tex.ac.uk/cgi-bin/texfaq2html?label=latexwords
% you have to redefine the macros for the
% language you use, e.g., american, ngerman
% (as chosen when loading babel/AtBeginDocument)
% ********************************************************************
\makeatletter
\@ifpackageloaded{babel}%
  {%
    \addto\extrasamerican{%
      \renewcommand*{\figureautorefname}{Figure}%
      \renewcommand*{\tableautorefname}{Table}%
      \renewcommand*{\partautorefname}{Part}%
      \renewcommand*{\chapterautorefname}{Chapter}%
      \renewcommand*{\sectionautorefname}{Section}%
      \renewcommand*{\subsectionautorefname}{Section}%
      \renewcommand*{\subsubsectionautorefname}{Section}%
    }%
    \addto\extrasngerman{%
      \renewcommand*{\paragraphautorefname}{Absatz}%
      \renewcommand*{\subparagraphautorefname}{Unterabsatz}%
      \renewcommand*{\footnoteautorefname}{Fu\"snote}%
      \renewcommand*{\FancyVerbLineautorefname}{Zeile}%
      \renewcommand*{\theoremautorefname}{Theorem}%
      \renewcommand*{\appendixautorefname}{Anhang}%
      \renewcommand*{\equationautorefname}{Gleichung}%
      \renewcommand*{\itemautorefname}{Punkt}%
    }%
      % Fix to getting autorefs for subfigures right (thanks to Belinda Vogt for changing the definition)
      \providecommand{\subfigureautorefname}{\figureautorefname}%
    }{\relax}
\makeatother

\listfiles
\linespread{1.1} % a bit more for Palatino
% ****************************************************************************************************
% If you like the classicthesis, then I would appreciate a postcard.
% My address can be found in the file ClassicThesis.pdf. A collection
% of the postcards I received so far is available online at
% http://postcards.miede.de
% ****************************************************************************************************

\RequirePackage{silence} % :-\suppress an unnecessary warning in compilation
    \WarningFilter{scrreprt}{Usage of package `titlesec'}
    \WarningFilter{titlesec}{Non standard sectioning command detected}
    \WarningFilter{hyperref}{Token not allowed in a PDF string (PDFDocEncoding)}
   
% ****************************************************************************************************
% 0. Set the encoding of your files. UTF-8 is the only sensible encoding nowadays. If you can't read
% äöüßáéçèê∂åëæƒÏ€ then change the encoding setting in your editor, not the line below. If your editor
% does not support utf8 use another editor!
% ****************************************************************************************************
\PassOptionsToPackage{utf8}{inputenc}
  \usepackage{inputenc}

\PassOptionsToPackage{T1}{fontenc} % T2A for cyrillics
  \usepackage{fontenc}

\PassOptionsToPackage{final}{microtype}  %jasine
\usepackage[final]{microtype} %jasine

\emergencystretch=1em

% *******************************************************************************************************************
% Note : Do not edit above this line
% Note : Do not edit options to classicthesis given below except the options drafting, printready and numsupervisors
% and eulermath
% 
% *******************************************************************************************************************
\PassOptionsToPackage{
   tocaligned=false, 
  dottedtoc=true,
  eulerchapternumbers=true, 
  linedheaders=false,      
  eulermath=true, %true, %false %recommended value is true. If false is used, cmmodern will be used.
  beramono=true,   
  palatino=true,    
  style=classicthesis, 
  floatperchapter=true,     % numbering per chapter for all floats (i.e., Figure 1.1)
  numsupervisors=one, %one %two %three %set according to your number of supervisors
  %drafting=true,    % print version information on the bottom of the pages only for draft
  drafting=false, % for final version this line is to be used
  printready=false %Enable this for making title page and hyperreflinks in colour for screen reading
  %printready=true  %Enable this for making title page and hyperreflinks in black for printing
 }{classicthesis}


% ****************************************************************************************************
% 2. Personal data and user ad-hoc commands (insert your own data here)
% ****************************************************************************************************

\newcommand{\myTitle}{Main Title of Your Thesis\xspace}
\newcommand{\mySubtitle}{Subtitle of Your Thesis\xspace}
\newcommand{\myDegree}{Doctor of Philosophy\xspace}
%\newcommand{\myDegree}{Master of Science \xspace}
%\newcommand{\myDegree}{Master of Science in Engineering \xspace}
\newcommand{\myName}{Your Full Name\xspace}
\newcommand{\myGender}{her\xspace}
%\newcommand{\myGender}{him\xspace}
\newcommand{\mySupervisorOne}{First Supervisor Name}
\newcommand{\mySupervisorTwo}{Second Supervisor Name}%used only if necessary
\newcommand{\mySupervisorThree}{Third Supervisor Name}%used only if necessary
\newcommand{\myDepartment}{Department of Computer Science and Engineering\xspace}
\newcommand{\myUni}{Indian Institute of Technology Palakkad\xspace}
\newcommand{\myLocation}{Palakkad\xspace}
\newcommand{\myTime}{Month YYYY\xspace}
\newcommand{\myVersion}{\classicthesis}
% ********************************************************************
% Setup, finetuning, and useful commands
% ********************************************************************
\providecommand{\mLyX}{L\kern-.1667em\lower.25em\hbox{Y}\kern-.125emX\@}
\newcommand{\ie}{i.\,e.}
\newcommand{\Ie}{I.\,e.}
\newcommand{\eg}{e.\,g.}
\newcommand{\Eg}{E.\,g.}
% ****************************************************************************************************


% ****************************************************************************************************
% 3. Loading some handy packages
% ****************************************************************************************************
% ********************************************************************
% Packages with options that might require adjustments
% ********************************************************************
\PassOptionsToPackage{ngerman,american}{babel} % change this to your language(s), main language last
\usepackage{babel}

\usepackage{csquotes}
%%%%%You may change the reference format to any of the options given agaist the parameters. 
%%%%%It is suggested that you maintain compatibility with natbib.
\PassOptionsToPackage{%
  %backend=biber,bibencoding=utf8, %instead of bibtex
  backend=bibtex8,bibencoding=ascii,%
  language=auto,%
  style=numeric-comp,%
  %style=authoryear-comp, % Author 1999, 2010
  %bibstyle=authoryear,dashed=false, % dashed: substitute rep. author with ---
  sorting=nyt, % name, year, title
  maxbibnames=10, % default: 3, et al.
  %backref=true,%
  natbib=true % natbib compatibility mode (\citep and \citet still work)
}{biblatex}
\usepackage{biblatex}

\PassOptionsToPackage{fleqn}{amsmath}       % math environments and more by the AMS
   \usepackage{amsmath}

\usepackage{amssymb}
\usepackage{wasysym}
\usepackage{amsthm} 
\usepackage{mathrsfs}
\usepackage{mathtools}
\usepackage{epsfig}
%%%%Theorem like environments follow chapterwise numbering, 
%%%%only exceptions being 'Remark' and 'Note'.
%%%%Do not edit this basic setting, if you are adding more such environments.
\theoremstyle{plain}% default
\newtheorem{theorem}{Theorem}[section]
\newtheorem{lemma}[theorem]{Lemma}
\newtheorem{proposition}[theorem]{Proposition}
\newtheorem{corollary}[theorem]{Corollary}
\newtheorem{property}[theorem]{Property}

\theoremstyle{definition}
\newtheorem{definition}[theorem]{Definition}
\newtheorem{conjecture}[theorem]{Conjecture}
\newtheorem{example}[theorem]{Example}
\newtheorem{observation}[theorem]{Observation}

\theoremstyle{remark}
\newtheorem*{remark}{Remark}
\newtheorem*{note}{Note}
\newtheorem{claim}{Claim}[chapter]

\usepackage[ruled,vlined,algochapter]{algorithm2e}

% ********************************************************************
% General useful packages
% ********************************************************************
\usepackage{graphicx} %
\usepackage{scrhack} % fix warnings when using KOMA with listings package
\usepackage{xspace} % to get the spacing after macros right
\PassOptionsToPackage{printonlyused,smaller}{acronym}
  \usepackage{acronym} % nice macros for handling all acronyms in the thesis
  \def\bflabel#1{{\acsfont{#1}\hfill}}
  \def\aclabelfont#1{\acsfont{#1}}
% ****************************************************************************************************
%\usepackage{pgfplots} % External TikZ/PGF support (thanks to Andreas Nautsch)
%\usetikzlibrary{external}
%\tikzexternalize[mode=list and make, prefix=ext-tikz/]
% ****************************************************************************************************




% ****************************************************************************************************
% 4. Setup floats: tables, (sub)figures, and captions
%%%Do not edit this setting
% ****************************************************************************************************
\usepackage{tabularx} % better tables
  \setlength{\extrarowheight}{3pt} % increase table row height
\newcommand{\tableheadline}[1]{\multicolumn{1}{l}{\spacedlowsmallcaps{#1}}}
\newcommand{\myfloatalign}{\centering} % to be used with each float for alignment
\usepackage{subfig}
% ****************************************************************************************************


% ****************************************************************************************************
% 5. Setup code listings
%%Do not edit this setting
% ****************************************************************************************************
\usepackage{listings}
%\lstset{emph={trueIndex,root},emphstyle=\color{BlueViolet}}%\underbar} % for special keywords
\lstset{language=[LaTeX]Tex,%C++,
  morekeywords={PassOptionsToPackage,selectlanguage},
  keywordstyle=\color{CTkeyword},%\bfseries,
  basicstyle=\small\ttfamily,
  %identifierstyle=\color{NavyBlue},
  commentstyle=\color{CTcomment}\ttfamily,
  stringstyle=\rmfamily,
  numbers=none,%left,%
  numberstyle=\scriptsize,%\tiny
  stepnumber=5,
  numbersep=8pt,
  showstringspaces=false,
  breaklines=true,
  %frameround=ftff,
  %frame=single,
  belowcaptionskip=.75\baselineskip
  %frame=L
}
% ****************************************************************************************************

%%%%Note that for list of symbols to be inluded the following commands needs to be executed
%%%%pdflatex ClassicThesis.tex
%%%%makeindex ClassicThesis.nlo -s nomencl.ist -o ClassicThesis.nls
%%%%pdflatex ClassicThesis.tex
\usepackage{nomencl}
\makenomenclature
\renewcommand{\nomname}{List of Symbols}
% ****************************************************************************************************
% 6. Last calls before the bar closes
% ****************************************************************************************************
% ********************************************************************
% Her Majesty herself
% ********************************************************************
\usepackage{classicthesis}


% ********************************************************************
% Fine-tune hyperreferences (hyperref should be called last)
%%%Do not edit this setting
% ********************************************************************
\hypersetup{%
  colorlinks=true, linktocpage=true, pdfstartpage=3, pdfstartview=FitV,%
  breaklinks=true, pageanchor=true,%
  pdfpagemode=UseNone, %
  plainpages=false, bookmarksnumbered, bookmarksopen=true, bookmarksopenlevel=1,%
  hypertexnames=true, pdfhighlight=/O,%
  urlcolor=CTurl, linkcolor=CTlink, citecolor=CTcitation, %
  pdftitle={\myTitle},%
  pdfauthor={\textcopyright\ \myName, \myUni},%
  pdfsubject={},%
  pdfkeywords={},%
  pdfcreator={pdfLaTeX},%
  pdfproducer={LaTeX with hyperref and classicthesis}%
}

%%%%%%%%%Do not edit below this line %%%%%%%%%%%%%%%%%%%%%%%%%%%%%%%%%

% ********************************************************************
% Setup autoreferences (hyperref and babel)
% ********************************************************************
% There are some issues regarding autorefnames
% http://www.tex.ac.uk/cgi-bin/texfaq2html?label=latexwords
% you have to redefine the macros for the
% language you use, e.g., american, ngerman
% (as chosen when loading babel/AtBeginDocument)
% ********************************************************************
\makeatletter
\@ifpackageloaded{babel}%
  {%
    \addto\extrasamerican{%
      \renewcommand*{\figureautorefname}{Figure}%
      \renewcommand*{\tableautorefname}{Table}%
      \renewcommand*{\partautorefname}{Part}%
      \renewcommand*{\chapterautorefname}{Chapter}%
      \renewcommand*{\sectionautorefname}{Section}%
      \renewcommand*{\subsectionautorefname}{Section}%
      \renewcommand*{\subsubsectionautorefname}{Section}%
    }%
    \addto\extrasngerman{%
      \renewcommand*{\paragraphautorefname}{Absatz}%
      \renewcommand*{\subparagraphautorefname}{Unterabsatz}%
      \renewcommand*{\footnoteautorefname}{Fu\"snote}%
      \renewcommand*{\FancyVerbLineautorefname}{Zeile}%
      \renewcommand*{\theoremautorefname}{Theorem}%
      \renewcommand*{\appendixautorefname}{Anhang}%
      \renewcommand*{\equationautorefname}{Gleichung}%
      \renewcommand*{\itemautorefname}{Punkt}%
    }%
      % Fix to getting autorefs for subfigures right (thanks to Belinda Vogt for changing the definition)
      \providecommand{\subfigureautorefname}{\figureautorefname}%
    }{\relax}
\makeatother

\listfiles
\linespread{1.1} % a bit more for Palatino
% ****************************************************************************************************
% If you like the classicthesis, then I would appreciate a postcard.
% My address can be found in the file ClassicThesis.pdf. A collection
% of the postcards I received so far is available online at
% http://postcards.miede.de
% ****************************************************************************************************

\RequirePackage{silence} % :-\suppress an unnecessary warning in compilation
    \WarningFilter{scrreprt}{Usage of package `titlesec'}
    \WarningFilter{titlesec}{Non standard sectioning command detected}
    \WarningFilter{hyperref}{Token not allowed in a PDF string (PDFDocEncoding)}
   
% ****************************************************************************************************
% 0. Set the encoding of your files. UTF-8 is the only sensible encoding nowadays. If you can't read
% äöüßáéçèê∂åëæƒÏ€ then change the encoding setting in your editor, not the line below. If your editor
% does not support utf8 use another editor!
% ****************************************************************************************************
\PassOptionsToPackage{utf8}{inputenc}
  \usepackage{inputenc}

\PassOptionsToPackage{T1}{fontenc} % T2A for cyrillics
  \usepackage{fontenc}

\PassOptionsToPackage{final}{microtype}  %jasine
\usepackage[final]{microtype} %jasine

\emergencystretch=1em

% *******************************************************************************************************************
% Note : Do not edit above this line
% Note : Do not edit options to classicthesis given below except the options drafting, printready and numsupervisors
% and eulermath
% 
% *******************************************************************************************************************
\PassOptionsToPackage{
   tocaligned=false, 
  dottedtoc=true,
  eulerchapternumbers=true, 
  linedheaders=false,      
  eulermath=true, %true, %false %recommended value is true. If false is used, cmmodern will be used.
  beramono=true,   
  palatino=true,    
  style=classicthesis, 
  floatperchapter=true,     % numbering per chapter for all floats (i.e., Figure 1.1)
  numsupervisors=one, %one %two %three %set according to your number of supervisors
  %drafting=true,    % print version information on the bottom of the pages only for draft
  drafting=false, % for final version this line is to be used
  printready=false %Enable this for making title page and hyperreflinks in colour for screen reading
  %printready=true  %Enable this for making title page and hyperreflinks in black for printing
 }{classicthesis}


% ****************************************************************************************************
% 2. Personal data and user ad-hoc commands (insert your own data here)
% ****************************************************************************************************

\newcommand{\myTitle}{Main Title of Your Thesis\xspace}
\newcommand{\mySubtitle}{Subtitle of Your Thesis\xspace}
\newcommand{\myDegree}{Doctor of Philosophy\xspace}
%\newcommand{\myDegree}{Master of Science \xspace}
%\newcommand{\myDegree}{Master of Science in Engineering \xspace}
\newcommand{\myName}{Your Full Name\xspace}
\newcommand{\myGender}{her\xspace}
%\newcommand{\myGender}{him\xspace}
\newcommand{\mySupervisorOne}{First Supervisor Name}
\newcommand{\mySupervisorTwo}{Second Supervisor Name}%used only if necessary
\newcommand{\mySupervisorThree}{Third Supervisor Name}%used only if necessary
\newcommand{\myDepartment}{Department of Computer Science and Engineering\xspace}
\newcommand{\myUni}{Indian Institute of Technology Palakkad\xspace}
\newcommand{\myLocation}{Palakkad\xspace}
\newcommand{\myTime}{Month YYYY\xspace}
\newcommand{\myVersion}{\classicthesis}
% ********************************************************************
% Setup, finetuning, and useful commands
% ********************************************************************
\providecommand{\mLyX}{L\kern-.1667em\lower.25em\hbox{Y}\kern-.125emX\@}
\newcommand{\ie}{i.\,e.}
\newcommand{\Ie}{I.\,e.}
\newcommand{\eg}{e.\,g.}
\newcommand{\Eg}{E.\,g.}
% ****************************************************************************************************


% ****************************************************************************************************
% 3. Loading some handy packages
% ****************************************************************************************************
% ********************************************************************
% Packages with options that might require adjustments
% ********************************************************************
\PassOptionsToPackage{ngerman,american}{babel} % change this to your language(s), main language last
\usepackage{babel}

\usepackage{csquotes}
%%%%%You may change the reference format to any of the options given agaist the parameters. 
%%%%%It is suggested that you maintain compatibility with natbib.
\PassOptionsToPackage{%
  %backend=biber,bibencoding=utf8, %instead of bibtex
  backend=bibtex8,bibencoding=ascii,%
  language=auto,%
  style=numeric-comp,%
  %style=authoryear-comp, % Author 1999, 2010
  %bibstyle=authoryear,dashed=false, % dashed: substitute rep. author with ---
  sorting=nyt, % name, year, title
  maxbibnames=10, % default: 3, et al.
  %backref=true,%
  natbib=true % natbib compatibility mode (\citep and \citet still work)
}{biblatex}
\usepackage{biblatex}

\PassOptionsToPackage{fleqn}{amsmath}       % math environments and more by the AMS
   \usepackage{amsmath}

\usepackage{amssymb}
\usepackage{wasysym}
\usepackage{amsthm} 
\usepackage{mathrsfs}
\usepackage{mathtools}
\usepackage{epsfig}
%%%%Theorem like environments follow chapterwise numbering, 
%%%%only exceptions being 'Remark' and 'Note'.
%%%%Do not edit this basic setting, if you are adding more such environments.
\theoremstyle{plain}% default
\newtheorem{theorem}{Theorem}[section]
\newtheorem{lemma}[theorem]{Lemma}
\newtheorem{proposition}[theorem]{Proposition}
\newtheorem{corollary}[theorem]{Corollary}
\newtheorem{property}[theorem]{Property}

\theoremstyle{definition}
\newtheorem{definition}[theorem]{Definition}
\newtheorem{conjecture}[theorem]{Conjecture}
\newtheorem{example}[theorem]{Example}
\newtheorem{observation}[theorem]{Observation}

\theoremstyle{remark}
\newtheorem*{remark}{Remark}
\newtheorem*{note}{Note}
\newtheorem{claim}{Claim}[chapter]

\usepackage[ruled,vlined,algochapter]{algorithm2e}

% ********************************************************************
% General useful packages
% ********************************************************************
\usepackage{graphicx} %
\usepackage{scrhack} % fix warnings when using KOMA with listings package
\usepackage{xspace} % to get the spacing after macros right
\PassOptionsToPackage{printonlyused,smaller}{acronym}
  \usepackage{acronym} % nice macros for handling all acronyms in the thesis
  \def\bflabel#1{{\acsfont{#1}\hfill}}
  \def\aclabelfont#1{\acsfont{#1}}
% ****************************************************************************************************
%\usepackage{pgfplots} % External TikZ/PGF support (thanks to Andreas Nautsch)
%\usetikzlibrary{external}
%\tikzexternalize[mode=list and make, prefix=ext-tikz/]
% ****************************************************************************************************




% ****************************************************************************************************
% 4. Setup floats: tables, (sub)figures, and captions
%%%Do not edit this setting
% ****************************************************************************************************
\usepackage{tabularx} % better tables
  \setlength{\extrarowheight}{3pt} % increase table row height
\newcommand{\tableheadline}[1]{\multicolumn{1}{l}{\spacedlowsmallcaps{#1}}}
\newcommand{\myfloatalign}{\centering} % to be used with each float for alignment
\usepackage{subfig}
% ****************************************************************************************************


% ****************************************************************************************************
% 5. Setup code listings
%%Do not edit this setting
% ****************************************************************************************************
\usepackage{listings}
%\lstset{emph={trueIndex,root},emphstyle=\color{BlueViolet}}%\underbar} % for special keywords
\lstset{language=[LaTeX]Tex,%C++,
  morekeywords={PassOptionsToPackage,selectlanguage},
  keywordstyle=\color{CTkeyword},%\bfseries,
  basicstyle=\small\ttfamily,
  %identifierstyle=\color{NavyBlue},
  commentstyle=\color{CTcomment}\ttfamily,
  stringstyle=\rmfamily,
  numbers=none,%left,%
  numberstyle=\scriptsize,%\tiny
  stepnumber=5,
  numbersep=8pt,
  showstringspaces=false,
  breaklines=true,
  %frameround=ftff,
  %frame=single,
  belowcaptionskip=.75\baselineskip
  %frame=L
}
% ****************************************************************************************************

%%%%Note that for list of symbols to be inluded the following commands needs to be executed
%%%%pdflatex ClassicThesis.tex
%%%%makeindex ClassicThesis.nlo -s nomencl.ist -o ClassicThesis.nls
%%%%pdflatex ClassicThesis.tex
\usepackage{nomencl}
\makenomenclature
\renewcommand{\nomname}{List of Symbols}
% ****************************************************************************************************
% 6. Last calls before the bar closes
% ****************************************************************************************************
% ********************************************************************
% Her Majesty herself
% ********************************************************************
\usepackage{classicthesis}


% ********************************************************************
% Fine-tune hyperreferences (hyperref should be called last)
%%%Do not edit this setting
% ********************************************************************
\hypersetup{%
  colorlinks=true, linktocpage=true, pdfstartpage=3, pdfstartview=FitV,%
  breaklinks=true, pageanchor=true,%
  pdfpagemode=UseNone, %
  plainpages=false, bookmarksnumbered, bookmarksopen=true, bookmarksopenlevel=1,%
  hypertexnames=true, pdfhighlight=/O,%
  urlcolor=CTurl, linkcolor=CTlink, citecolor=CTcitation, %
  pdftitle={\myTitle},%
  pdfauthor={\textcopyright\ \myName, \myUni},%
  pdfsubject={},%
  pdfkeywords={},%
  pdfcreator={pdfLaTeX},%
  pdfproducer={LaTeX with hyperref and classicthesis}%
}

%%%%%%%%%Do not edit below this line %%%%%%%%%%%%%%%%%%%%%%%%%%%%%%%%%

% ********************************************************************
% Setup autoreferences (hyperref and babel)
% ********************************************************************
% There are some issues regarding autorefnames
% http://www.tex.ac.uk/cgi-bin/texfaq2html?label=latexwords
% you have to redefine the macros for the
% language you use, e.g., american, ngerman
% (as chosen when loading babel/AtBeginDocument)
% ********************************************************************
\makeatletter
\@ifpackageloaded{babel}%
  {%
    \addto\extrasamerican{%
      \renewcommand*{\figureautorefname}{Figure}%
      \renewcommand*{\tableautorefname}{Table}%
      \renewcommand*{\partautorefname}{Part}%
      \renewcommand*{\chapterautorefname}{Chapter}%
      \renewcommand*{\sectionautorefname}{Section}%
      \renewcommand*{\subsectionautorefname}{Section}%
      \renewcommand*{\subsubsectionautorefname}{Section}%
    }%
    \addto\extrasngerman{%
      \renewcommand*{\paragraphautorefname}{Absatz}%
      \renewcommand*{\subparagraphautorefname}{Unterabsatz}%
      \renewcommand*{\footnoteautorefname}{Fu\"snote}%
      \renewcommand*{\FancyVerbLineautorefname}{Zeile}%
      \renewcommand*{\theoremautorefname}{Theorem}%
      \renewcommand*{\appendixautorefname}{Anhang}%
      \renewcommand*{\equationautorefname}{Gleichung}%
      \renewcommand*{\itemautorefname}{Punkt}%
    }%
      % Fix to getting autorefs for subfigures right (thanks to Belinda Vogt for changing the definition)
      \providecommand{\subfigureautorefname}{\figureautorefname}%
    }{\relax}
\makeatother

\listfiles
\linespread{1.1} % a bit more for Palatino
% ****************************************************************************************************
% If you like the classicthesis, then I would appreciate a postcard.
% My address can be found in the file ClassicThesis.pdf. A collection
% of the postcards I received so far is available online at
% http://postcards.miede.de
% ****************************************************************************************************

\RequirePackage{silence} % :-\suppress an unnecessary warning in compilation
    \WarningFilter{scrreprt}{Usage of package `titlesec'}
    \WarningFilter{titlesec}{Non standard sectioning command detected}
    \WarningFilter{hyperref}{Token not allowed in a PDF string (PDFDocEncoding)}
   
% ****************************************************************************************************
% 0. Set the encoding of your files. UTF-8 is the only sensible encoding nowadays. If you can't read
% äöüßáéçèê∂åëæƒÏ€ then change the encoding setting in your editor, not the line below. If your editor
% does not support utf8 use another editor!
% ****************************************************************************************************
\PassOptionsToPackage{utf8}{inputenc}
  \usepackage{inputenc}

\PassOptionsToPackage{T1}{fontenc} % T2A for cyrillics
  \usepackage{fontenc}

\PassOptionsToPackage{final}{microtype}  %jasine
\usepackage[final]{microtype} %jasine

\emergencystretch=1em

% *******************************************************************************************************************
% Note : Do not edit above this line
% Note : Do not edit options to classicthesis given below except the options eulermath, drafting, printready and numsupervisors
% % 
% *******************************************************************************************************************
\PassOptionsToPackage{
   tocaligned=false, 
  dottedtoc=true,
  eulerchapternumbers=true, 
  linedheaders=false,      
  eulermath=true, %true, %false %If false is set, cmmodern font will be used for math. %If text font is changed, be careful about this option.  
  beramono=true,   
  palatino=true,    
  style=classicthesis, 
  floatperchapter=true,     % numbering per chapter for all floats (i.e., Figure 1.1)
  numsupervisors=one, %one %two %three %set according to your number of supervisors
  %drafting=true,    % print version information on the bottom of the pages only for draft
  drafting=false, % for final version this line is to be used
  %printready=false %Enable this for making title page and hyperreflinks in colour for screen reading
  printready=true  %Enable this for making title page and hyperreflinks in black for printing
 }{classicthesis}


% ****************************************************************************************************
% 2. Personal data and user ad-hoc commands (insert your own data here)
% ****************************************************************************************************

\newcommand{\myTitle}{Main Title of Your Thesis\xspace}
%\newcommand{\mySubtitle}{Subtitle of Your Thesis\xspace}
\newcommand{\myDegree}{Doctor of Philosophy\xspace}
%\newcommand{\myDegree}{Master of Science \xspace}
%\newcommand{\myDegree}{Master of Technology \xspace}
%\newcommand{\myDegree}{Master of Science in Engineering \xspace}
\newcommand{\myName}{Your Full Name\xspace}
\newcommand{\myRollNo}{YourRollNo\xspace}
\newcommand{\myGender}{her\xspace}
%\newcommand{\myGender}{him\xspace}
\newcommand{\mySupervisorOne}{Name of Supervisor}
\newcommand{\mySupervisorTwo}{Name of Supervisor2}%used only if necessary
\newcommand{\mySupervisorThree}{Name of Supervisor3}%used only if necessary
\newcommand{\myDepartment}{Department of Computer Science and Engineering\xspace}
\newcommand{\myUni}{Indian Institute of Technology Palakkad\xspace}
\newcommand{\myLocation}{Palakkad\xspace}
\newcommand{\myTime}{Month YYYY\xspace}
\newcommand{\myVersion}{\classicthesis}
% ********************************************************************
% Setup, finetuning, and useful commands
% ********************************************************************
\providecommand{\mLyX}{L\kern-.1667em\lower.25em\hbox{Y}\kern-.125emX\@}
\newcommand{\ie}{i.\,e.}
\newcommand{\Ie}{I.\,e.}
\newcommand{\eg}{e.\,g.}
\newcommand{\Eg}{E.\,g.}
% ****************************************************************************************************


% ****************************************************************************************************
% 3. Loading some handy packages
% ****************************************************************************************************
% ********************************************************************
% Packages with options that might require adjustments
% ********************************************************************
\PassOptionsToPackage{ngerman,american}{babel} % change this to your language(s), main language last
\usepackage{babel}

\usepackage{csquotes}
%%%%%You may change the reference format to any of the options given agaist the parameters. 
%%%%%It is suggested that you maintain compatibility with natbib.
\PassOptionsToPackage{%
  %backend=biber,bibencoding=utf8, %instead of bibtex
  backend=bibtex8,bibencoding=ascii,%
  language=auto,%
  style=numeric-comp,%
  %style=authoryear-comp, % Author 1999, 2010
  %bibstyle=authoryear,dashed=false, % dashed: substitute rep. author with ---
  sorting=nyt, % name, year, title
  maxbibnames=10, % default: 3, et al.
  %backref=true,%
  natbib=true % natbib compatibility mode (\citep and \citet still work)
}{biblatex}
\usepackage{biblatex}

\PassOptionsToPackage{fleqn}{amsmath}       % math environments and more by the AMS
   \usepackage{amsmath}

\usepackage{amssymb}
\usepackage{wasysym}
\usepackage{amsthm} 
\usepackage{mathrsfs}
\usepackage{mathtools}
\usepackage{epsfig}
%%%%Theorem like environments follow sectionwise numbering, 
%%%%only exceptions being 'Remark' and 'Note'.
%%%%Do not edit this basic setting, if you are adding more such environments.
\theoremstyle{plain}% default
\newtheorem{theorem}{Theorem}[section]
\newtheorem{lemma}[theorem]{Lemma}
\newtheorem{proposition}[theorem]{Proposition}
\newtheorem{corollary}[theorem]{Corollary}
\newtheorem{property}[theorem]{Property}

\theoremstyle{definition}
\newtheorem{definition}[theorem]{Definition}
\newtheorem{conjecture}[theorem]{Conjecture}
\newtheorem{example}[theorem]{Example}
\newtheorem{observation}[theorem]{Observation}

\theoremstyle{remark}
\newtheorem*{remark}{Remark}
\newtheorem*{note}{Note}
\newtheorem{claim}{Claim}[chapter]

\usepackage[ruled,vlined,algochapter]{algorithm2e}

% ********************************************************************
% General useful packages
% ********************************************************************
\usepackage{graphicx} %
\usepackage{scrhack} % fix warnings when using KOMA with listings package
\usepackage{xspace} % to get the spacing after macros right
\PassOptionsToPackage{printonlyused,smaller}{acronym}
  \usepackage{acronym} % nice macros for handling all acronyms in the thesis
  \def\bflabel#1{{\acsfont{#1}\hfill}}
  \def\aclabelfont#1{\acsfont{#1}}
% ****************************************************************************************************
%\usepackage{pgfplots} % External TikZ/PGF support (thanks to Andreas Nautsch)
%\usetikzlibrary{external}
%\tikzexternalize[mode=list and make, prefix=ext-tikz/]
% ****************************************************************************************************




% ****************************************************************************************************
% 4. Setup floats: tables, (sub)figures, and captions
%%%Do not edit this setting
% ****************************************************************************************************
\usepackage{tabularx} % better tables
  \setlength{\extrarowheight}{3pt} % increase table row height
\newcommand{\tableheadline}[1]{\multicolumn{1}{l}{\spacedlowsmallcaps{#1}}}
\newcommand{\myfloatalign}{\centering} % to be used with each float for alignment
\usepackage{subfig}
% ****************************************************************************************************


% ****************************************************************************************************
% 5. Setup code listings
%%Do not edit this setting
% ****************************************************************************************************
\usepackage{listings}
%\lstset{emph={trueIndex,root},emphstyle=\color{BlueViolet}}%\underbar} % for special keywords
\lstset{language=[LaTeX]Tex,%C++,
  morekeywords={PassOptionsToPackage,selectlanguage},
  keywordstyle=\color{CTkeyword},%\bfseries,
  basicstyle=\small\ttfamily,
  %identifierstyle=\color{NavyBlue},
  commentstyle=\color{CTcomment}\ttfamily,
  stringstyle=\rmfamily,
  numbers=none,%left,%
  numberstyle=\scriptsize,%\tiny
  stepnumber=5,
  numbersep=8pt,
  showstringspaces=false,
  breaklines=true,
  %frameround=ftff,
  %frame=single,
  belowcaptionskip=.75\baselineskip
  %frame=L
}
% ****************************************************************************************************

%%%%Note that for list of symbols to be inluded the following commands needs to be executed
%%%%pdflatex ClassicThesis.tex
%%%%makeindex ClassicThesis.nlo -s nomencl.ist -o ClassicThesis.nls
%%%%pdflatex ClassicThesis.tex
\usepackage{nomencl}
\makenomenclature
\renewcommand{\nomname}{List of Symbols}
\setlength{\nomlabelwidth}{2.5cm}
% ****************************************************************************************************
% 6. Last calls before the bar closes
% ****************************************************************************************************
% ********************************************************************
% Her Majesty herself
% ********************************************************************
\usepackage{classicthesis}


% ********************************************************************
% Fine-tune hyperreferences (hyperref should be called last)
%%%Do not edit this setting
% ********************************************************************
\hypersetup{%
  colorlinks=true, linktocpage=true, pdfstartpage=3, pdfstartview=FitV,%
  breaklinks=true, pageanchor=true,%
  pdfpagemode=UseNone, %
  plainpages=false, bookmarksnumbered, bookmarksopen=true, bookmarksopenlevel=1,%
  hypertexnames=true, pdfhighlight=/O,%
  urlcolor=CTurl, linkcolor=CTlink, citecolor=CTcitation, %
  pdftitle={\myTitle},%
  pdfauthor={\textcopyright\ \myName, \myUni},%
  pdfsubject={},%
  pdfkeywords={},%
  pdfcreator={pdfLaTeX},%
  pdfproducer={LaTeX with hyperref and classicthesis}%
}

%%%%%%%%%Do not edit below this line %%%%%%%%%%%%%%%%%%%%%%%%%%%%%%%%%

% ********************************************************************
% Setup autoreferences (hyperref and babel)
% ********************************************************************
% There are some issues regarding autorefnames
% http://www.tex.ac.uk/cgi-bin/texfaq2html?label=latexwords
% you have to redefine the macros for the
% language you use, e.g., american, ngerman
% (as chosen when loading babel/AtBeginDocument)
% ********************************************************************
\makeatletter
\@ifpackageloaded{babel}%
  {%
    \addto\extrasamerican{%
      \renewcommand*{\figureautorefname}{Figure}%
      \renewcommand*{\tableautorefname}{Table}%
      \renewcommand*{\partautorefname}{Part}%
      \renewcommand*{\chapterautorefname}{Chapter}%
      \renewcommand*{\sectionautorefname}{Section}%
      \renewcommand*{\subsectionautorefname}{Section}%
      \renewcommand*{\subsubsectionautorefname}{Section}%
    }%
    \addto\extrasngerman{%
      \renewcommand*{\paragraphautorefname}{Absatz}%
      \renewcommand*{\subparagraphautorefname}{Unterabsatz}%
      \renewcommand*{\footnoteautorefname}{Fu\"snote}%
      \renewcommand*{\FancyVerbLineautorefname}{Zeile}%
      \renewcommand*{\theoremautorefname}{Theorem}%
      \renewcommand*{\appendixautorefname}{Anhang}%
      \renewcommand*{\equationautorefname}{Gleichung}%
      \renewcommand*{\itemautorefname}{Punkt}%
    }%
      % Fix to getting autorefs for subfigures right (thanks to Belinda Vogt for changing the definition)
      \providecommand{\subfigureautorefname}{\figureautorefname}%
    }{\relax}
\makeatother

\listfiles
\linespread{1.1} % a bit more for Palatino
