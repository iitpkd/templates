%*******************************************************
% Abstract
%*******************************************************
%\renewcommand{\abstractname}{Abstract}
\pdfbookmark[1]{Abstract}{Abstract}
% \addcontentsline{toc}{chapter}{\tocEntry{Abstract}}
\begingroup
\let\clearpage\relax
\let\cleardoublepage\relax
\let\cleardoublepage\relax
%%%%**********************************************************************************
%%% Note: Do not edit above this line
%%%%**********************************************************************************
\chapter*{Abstract}
Add abstract of your thesis here. 
a great guide by
Kent Beck how to write good abstracts can be found here:
\begin{center}
\url{https://plg.uwaterloo.ca/~migod/research/beckOOPSLA.html}
\end{center}
This thesis mainly focuses on algorithmic and combinatorial questions related to some geometric problems on graphs. 
In the last part of this thesis, a graph coloring problem is also discussed.
\paragraph{\textbf{Topic1 summary:}}
These are graph parameters dealing with geometric representations of graphs in higher dimensions. 
Both these parameters are known to be NP-Hard to compute in general and are even hard to approximate within an $O(n^{1-\epsilon})$ factor for any $\epsilon >0$, 
under standard complexity theoretic assumptions. 

We studied algorithmic questions for these problems, for certain graph classes, to yield efficient algorithms or approximations. 
Our results include a polynomial time constant factor approximation algorithm for computing the cubicity of trees and a polynomial time
constant ($\le 2.5$) factor approximation algorithm for computing the boxicity of circular arc graphs. As far as we know, there were no constant
factor approximation algorithms known previously, for computing boxicity or cubicity of any well known graph class for which the respective parameter 
value is unbounded.
\paragraph{\textbf{Topic2 summary:}}
A graph is outerplanar, if it has a planar embedding with all its vertices lying on the outer face. 
We give an efficient algorithm to $2$-vertex-connect any connected outerplanar graph $G$ by adding more edges to it, 
in order to obtain a supergraph of $G$ such that the resultant graph is still outerplanar and its pathwidth is within a constant times 
the pathwidth of $G$. 
This algorithm leads to a constant factor approximation algorithm for computing minimum height planar straight 
line grid-drawings of outerplanar graphs, extending the existing algorithm known for $2$-vertex connected outerplanar graphs.

\bigskip
\noindent
{\textbf{Keywords:} Keyword One, Keyword Two, Keyword Three}\\
\smallskip
\noindent
{\textbf{AMS Subject Classification:} Give Class here . Remove if irrelevant}\\
\smallskip
\noindent
{\textbf{ACM Subject Classification:} Give Class here . Remove if irrelevant}
\endgroup

\vfill
