\documentclass[11pt,paper=a4,answers]{exam}
\usepackage{graphicx,lastpage}
\usepackage{upgreek}
\usepackage{censor}
\usepackage{amsmath}
\usepackage{listings}
\usepackage{amsfonts}
\usepackage{enumerate}
\censorruledepth=-.2ex
\censorruleheight=.1ex

\hyphenpenalty 10000
\usepackage[paperheight=10.5in,paperwidth=8.27in,bindingoffset=0in,left=0.8in,right=1in,
top=0.7in,bottom=1in,headsep=.5\baselineskip]{geometry}
\flushbottom

\usepackage[normalem]{ulem}
\renewcommand\ULthickness{2pt}   %%---> For changing thickness of underline
\setlength\ULdepth{1.5ex}%\maxdimen ---> For changing depth of underline
\renewcommand{\baselinestretch}{1}
\pagestyle{empty}

\pagestyle{headandfoot}
\headrule
\newcommand{\continuedmessage}{%
	\ifcontinuation{\footnotesize Question \ContinuedQuestion\ continues\ldots}{}%
}
\runningheader{\footnotesize CSE}
{\footnotesize Quiz 4}
{\footnotesize Page \thepage\ of \numpages}
\footrule
\footer{\footnotesize}
{}
{\ifincomplete{\footnotesize Question \IncompleteQuestion\ continues
		on the next page\ldots}{\iflastpage{\footnotesize End of exam}{\footnotesize Continued on next page\ldots}}}

\usepackage{cleveref}
\crefname{figure}{figure}{figures}
\crefname{question}{question}{questions}
%==============================================================
\begin{document}
	
	%% \thispagestyle{empty}
	
	\noindent
	\begin{minipage}[l]{.14\textwidth}%
		\noindent
		\includegraphics[scale=0.3]{logo}
	\end{minipage}
	\hfill
	\begin{minipage}[r]{.65\textwidth}%
		\begin{center}
			{\large \bfseries COMPUTER SCIENCE AND ENGINEERING \par
				\large Indian Institute of Technology, Palakkad \\[2pt]
				Course Code: Course Name\\
				\normalsize\emph{\underline{Test I (11 September, 2020)}}
			}
			%  \vspace{0.5cm}
		\end{center}
	\end{minipage}
	\hfill
	\begin{minipage}[r]{.14\textwidth}%
		\noindent
		\begin{flushright}
			{\footnotesize }	
		\end{flushright}
	\end{minipage}
	\par
	\noindent
	\uline{ Time: 08:00 --- 09:00 hrs \hfill  Max Marks: 20 }
	
	
	\begin{questions}
		
		\pointsinrightmargin
		\pointsdroppedatright
		\marksnotpoints
		%\marginpointname{mark}
		\pointpoints{mark}{marks}
		\pointformat{\boldmath\themarginpoints}
		\bracketedpoints
		
		\question
 		This is the context of the first question
		
		\begin{parts}
			\part[10]
			This is the first part. \droppoints
			\part[4]
			This is the second part. \droppoints
		\end{parts}	

		
		\question
		Given an example for each of the following
		\begin{parts}
			\part[2]
			Part one. \droppoints
			
			\part[2]
			Part two. \droppoints
			
			\part[2]
			Part three. \droppoints
		\end{parts}

	\end{questions}
	
\end{document} 